\newpage
\section{Introducción}

    

%   	\LARGE \textbf{Introducción\\}
   	
   	%\vspace{5mm}
\begin{footnotesize}	
	El presente documento tiene como objeto poner a disposición de los interesados, el análisis de los procesos que se consideran primordiales en el funcionamiento del departamento de coordinación de educación continua:
\begin{itemize}
	\item[\textbf{P1}] Gestión de cursos
\end{itemize}
\vspace{1cm}
Haciendo uso del  Manual Administrativo de Aplicación General en las materias de Tecnologías de la Información, Comunicaciones y de Seguridad de la Información (\textbf{MAAGTICSI})  se definieron los siguientes apartados como relevantes para el análisis:\\

\textit{Objetivo General}: Establece el alcance y los principios del proceso. Responde a las preguntas de \textit{¿Qué?}, \textit{¿Dónde?}, \textit{¿Cuándo?}, \textit{¿Por qué?} y \textit{¿Para qué?}.\\

\textit{Objetivos Específicos}: Son objetivos que intervienen en el cumplimiento del objetivo general, inciden en metas indirectas con un alcance menor.\\

\textit{Descripción}: Es la explicación del desarrollo del proceso a grandes rasgos, es también, una guía general del flujo que sigue el proceso en circunstancias normales.\\

\textit{Mapa General del Proceso}: Es una representación gráfica del proceso usando la notación de Modelado de Procesos de Negocio (\textbf{BPMN}). En éste se observa el flujo que sigue el proceso en condiciones optimas.\\

\textit{Reglas del proceso}. Es una lista de todas aquellas reglas que se deben cumplir para que el proceso se considere correctamente realizado.\\

\textit{Roles del proceso}. Es la especificación de las personas o actores que participan en el proceso, éste listado incluye la descripción de la actividad que desempeña dentro del proceso.\\


\textit{Actividades del proceso}: Es la explicación de las actividades que conforman el proceso. Esta explicación incluye:
\begin{itemize}
\item \textit{Descripción}. Es la especificación de las tareas que se realizan en una actividad y en que consiste cada una de ellas.\\ 
\item \textit{Factores críticos}. Es un listado de todas las tareas y/o actividades que los roles desempeñan para que se efectúe el proceso, con el objeto que fue planteado.\\
\end{itemize}

\textit{Relación de Productos del proceso}. Es un listado de los productos que tienen relación como entrada o salida en el proceso, con su respectiva descripción.\\

\textit{Indicador del proceso}. Es una relación de aquellos parámetros que son medibles dentro del proceso. Incluye:
	\begin{itemize}
		\item Nombre
		\item Objetivo
		\item Descripción
		\item Dimensión. Refiere al aspecto que es medible por el indicador. Eficiencia, calidad, efectividad, etc.
		\item Fórmula. Es la forma de calcular dicho indicador.
		\item Responsable. Es la persona(s) encargada de realizar el cálculo del indicador.
		\item Frecuencia de cálculo. Es la periodicidad con la que el indicador debe ser medido.\\
	\end{itemize}

Finalmente se pretende que éste se convierta en un documento de consulta para cualquier implicado en el desarrollo del proyecto, permitiendo el profundo entendimiento de lo aquí detallado.

\end{footnotesize}

