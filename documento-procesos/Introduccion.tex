%%TAPIA MUJICA
\newpage
\section{Introducción}

    

%   	\LARGE \textbf{Introducción\\}
   	
   	%\vspace{5mm}
\begin{footnotesize}	
	El presente documento tiene como objetivo poner a disposición de los interesados el análisis de 5 procesos que se consideran primordiales en el funcionamiento del departamento de coordinación de educación continua:\\

\begin{enumerate}
\item Reinscripción
\end{enumerate}

Siguiendo el Manual Administrativo de Aplicación General en las materias de Tecnologías de la Información (\textbf{MAAGTIC}) se definieron los siguientes apartados como relevantes para el análisis:\\

\textit{Objetivo General}: Establece el alcance y los principios del proceso. Responde a las preguntas de \textit{¿Qué?}, \textit{¿Dónde?}, \textit{¿Cuándo?}, \textit{¿Por qué?} y \textit{¿Para qué?}.\\

\textit{Objetivos Específicos}: Son objetivos que intervienen en el cumplimiento del objetivo general, inciden en metas indirectas y de un alcance menor que el objetivo general.\\

\textit{Descripción}: Es la explicación del desarrollo del proceso a grandes rasgos, es, también, una guía general del flujo que sigue el proceso en circunstancias normales.\\

\textit{Mapa General del Proceso}: Es una representación gráfica del proceso usando la notación de Modelado de Procesos de Negocio (\textbf{BPMN}). En éste se observa el flujo que sigue el proceso desde que inicia hasta que termina.\\

\textit{Criterios de Entrada}: Son las condiciones que se deben cumplir para que el proceso pueda dar inicio.\\

\textit{Entradas}: Son todos los productos (documentos y/o datos) que provienen de una fuente externa al proceso pero que son importantes dentre de éste.\\

\textit{Criterios de Salida}: Son las condiciones que se deben cumplir para que el proceso pueda concluir.\\

\textit{Salidas}: Son todos los productos (documentos y/o datos) que se generan en el transcurso del proceso.\\

\textit{Descripción de las Actividades}: Es la explicación de las actividades que conforman el proceso. Esta explicación incluye:\\

\begin{itemize}
\item \textit{Descripción}. Es la especificación de las tareas que se realizan en una actividad y en que consiste cada una de ellas.\\ 
\item \textit{Precondiciones de Entrada}. Es un listado de las condiciones que debieron de haber sido satisfechas para poder dar inicio a la actividad.\\
\item \textit{Relación de Productos}. Es un listado de los productos que tienen relación como entrada o salida en la actividad.\\
\item \textit{Descripción de Roles}. Es la especificación de las personas o actores que participan en el proceso, éste listado incluye las tareas y responsabilidades que posee el mismo dentro del proceso.\\
\item \textit{Descripción de Productos}. Es un listado de aquellos productos que participan en el proceso.\\
\item \textit{Indicadores}. Es una relación de aquellos parámetros que son medibles dentro del proceso. Incluye:\\

	\begin{itemize}
		\item Nombre
		\item Objetivo
		\item Descripción
		\item Dimensión. Refiere al aspecto que es medible por el indicador. Eficiencia, calidad, efectividad, etc.
		\item Fórmula. Es la forma de calcular dicho indicador.
		\item Responsable. Es la persona(s) encargada de realizar el cálculo del indicador.
		\item Frecuencia de Cálculo. Es la periodicidad con la que el indicador debe ser medido.
	\end{itemize}

\item \textit{Reglas del Proceso}. Es una lista de todas aquellas reglas que se deben cumplir para que el proceso se considere correctamente realizado.\\
\item \textit{Factores Críticos}. Es un listado de todos aquellos posibles casos en los que el proceso pudiese verse afectado. Se incluye el impacto que éstos pueden tener en el flujo del proceso y se propone una solución desde un enfoque organizacional.\\
\end {itemize}

Finalmente se pretende que éste se convierta en un documento de consulta para cualquier implicado en el desarrollo del proyecto y que permita el profundo entendimiento de lo aquí detallado.

\end{footnotesize}

