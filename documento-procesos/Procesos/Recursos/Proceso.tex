%			IMPORTANTE			
%	Solo modifiquen la parte del id, fecha y area    
%		Lo demas dejarlo igual
\pagestyle{fancy}
\headheight=60pt 						
\fancyhead[L]
{	
	\begin{minipage}{2cm}
		\includegraphics[width=0.8\textwidth]{images/cec}\\
	\end{minipage}	
}

\fancyhead[C] 							
{
	\large			
	{
		Coordinaci�n de educaci�n continua\\%%Texto 
	}	
} 				

\fancyhead[R] 							
{
	\small
	{
		Proceso Id: 3\\ 	%Editar Id del proceso
		Fecha: 27/09/2014	\\	%Editar fecha 
		Nombre: Administraci�n de Recursos%Editar �rea
	}
}
 %Encabezado del documento, para editar ID del proceso, fecha y area deben usar este archivo
\begin{landscape}
  \begin{figure}[b]
  \section{Administraci�n de Recursos} 
	\hspace{4cm}
  \begin{center}
      \includegraphics[width=22cm]{images/Recursos}
       \caption{Diagrama que muestra la  interacci�n de las actividades en el proceso ''Administraci�n de Recursos''}
  \end{center}
 \end{figure}
\end{landscape}

\subsection{Objetivos del Proceso} 	%Objetivos del proceso
%[Definir los objetivos del proceso, en el contexto del funcionamiento del sistema (máximo 20 líneas). Y que responde a la siguiente
%pregunta ¿Cuál es la razón de ser este proceso? en otras palabras que hace y para que lo hace. 1 Objetivo general, n objetivos
% específicos].

\subsubsection{Objetivo General}	%Objetivos generales
Tiene como fin la planeación correspondiente al uso de las instalaciones de la escuela, considerando que el uso de las instalaciones no se limita solo como area de trabajo para estos cursos si no que además se tiene en gran medida el uso por parte de los proferos y los estudiantes de ESCOM durante el ciclo escolar.
\subsubsection{Objetivos Específicos}	%Objetivos especifico

\begin{enumerate}[I] 			%Empieza una lista numerada con los objetivos especificos.
	\item Tener una área de trabajo por curso, ya sea un laboratorio o salón, con el fin de permitir a los asistentes realizar las actividades que indica el curso.
	\item Que el área de trabajo destinada este disponible durante los horarios elaborados para cada curso.
	\item Contar con áreas de trabajo alternas, con la finalidad de alternativas en caso de que por razones ajenas a nuestra adinistración no se cuente de manera temporal o permanente con alguna de estas áreas.
\end{enumerate}

 	%En este archivo se encuentra lo relacionado con los objetivos del proceso, como son objetivo general y especificos
\subsection{Descripci�n del Proceso}
	El proceso da inicio al recibir la lista de los grupos �abiertos� y la definici�n del curso al que corresponde cada uno de los grupos, posteriormente se hace una petici�n de las aulas disponibles en la escuela; una vez con esta informaci�n se procede a ver si hay aulas que cubren con la disponibilidad del horario por los grupos, en caso de no haber disponibilidad se lleva a cabo un cambio de horario correspondiente al grupo afectado, adicionalmente para seleccionar el aula debe considerarse si el curso requiere equipo de c�mputo, as� como que hardware y software es necesario para trabajar. 
	
	Una vez que se selecciona un aula por grupo, se hace una solicitud del uso de la misma; una vez con la respuesta a esta solicitud, si es negativa se asigna una nueva aula, y si es afirmativa entonces se procede a notificar a la administraci�n de cursos que ya se cuenta con las aulas correspondientes.
	
	Por �ltimo, se hace una revisi�n del material did�ctico que ser� necesario como sistema operativo, IDE�s, software de apoyo, v�deo, archivos de texto, marcadores y de acuerdo al uso deseado se asignar� este al profesor, se instalar� y configurar� en el equipo de c�mputo, o bien se subir� a un servicio de almacenamiento en la nube, para que los alumnos tengan acceso a este. Al finalizar se hace notificaci�n a la administraci�n de Cursos que se cuenta con los recursos necesarios para dar inicio a los cursos.

\subsubsection{Criterios de Entrada}			%Criterios de entrada del proceso
%[Listar las condiciones que hacen que el proceso comience a trabajar].\\
	%Archivo para insertar los criterios de entrada
\subsubsection{Entradas}							%Entradas del proceso

\begin{table}[H]
\centering
	\begin{tabular}{p{5cm} p{11cm}}
		\hline
			\textbf{Entradas} & \textbf{Descripci�n}\\
		\hline
		\hline
		Horarios & Documento donde se especifica, las horas dedicadas por d�a de cada curso y su duraci�n aproximada de acuerdo al temario previsto. \\
		\hline
		 & .   \\	
		\hline
		 & .   \\
	    %		Copiar para agregar nueva fila con separando la entrada y la descripcion con &
	    %	\hline
	    %		Entrada N & Descripci�n de la entrada 2 \\	
	    %
		\hline
	\end{tabular}
\caption{Entradas de Nombre del Proceso}
\end{table}		%Archivo para insertar las entradas del proceso
\subsubsection{Criterios de Salida}			%Criterios de Salida del proceso
%[Listar las condiciones que hacen que el proceso comience a trabajar].\\

\begin{itemize}						%Inicia una lista con los criterios de salida
           \item ..
\end{itemize}	%Archivo para insertar los criterios de salida
\subsubsection{Salidas}

\begin{table}[H]
\centering
	\begin{tabular}{p{5cm} p{11cm}}
		\hline
			\textbf{Salidas} & \textbf{Descripción}\\
		\hline
		\hline
			Curso & Queda formalmente definido cuando quedan establecidos, revisados y en caso de ser necesario actualizados todos los temas que tienen que ser tratados a lo largo de la duraci\'on de este.
			\\
	
	    %		Copiar para agregar nueva fila con separando la entrada y la descripcion con &
	    %	\hline
	    %		Salida N & Descripción de la salida N \\	
	    %
		\hline	
	\end{tabular}
\caption{Entradas de Nombre del Proceso}
\end{table}
		%Archivo para insertar las salidas del proceso
\newpage
\subsubsection{Descripci�n de Actividades del Proceso}		%Descripcion de actividades del proceso



%%%%%%%%%%%%%%%%%%%%%%%%%%%%%%%%%%%% Actividades
\begin{table}[H]
\centering
	\begin{tabular}{p{5cm} p{11cm}}
		\hline
			\textbf{Proceso1-A1} & \textbf{Instalaciones}\\
		\hline\hline
			Descripci�n & 
			Una vez definida la cantidad de cursos que se impartiran durante el siguiente periodo, se establece cuantos salones y laboratorios, ser�n necesarios de acuerdo a la finalidad del curso, posteriormente se realiza una petici�n a Control Escolar de la disponibilidad de los salones y laboratorios de acuerdo a su uso de estos por parte de la ESCOM y otras instituciones que hagan uso de las instalaciones, en base a esta informaci�n se hace una primera selecci�n como propuesta para cada uno de los cursos, los cual se revisa y se le hacen modificaciones de acuerdo al horario deseado. 

			Una vez confirmado el horario en base a la disponibilidad de las instalaciones se procede a realizar un oficio solicitando a Control Escolar que se le facilite el uso de las �reas de trabajo seleccionadas para los cursos, una vez otorgado el permiso para el uso de las intalaciones por parte de la ESCOM se confirma la disponibilidad de las mismas para que se puedan publicar los horarios.

			Adicionalmente se guarda la informaci�n de los s�lones que no estan en uso durante los horarios establecidos, en caso de ser necesarios por razones que esten fuera del alcance de la administraci�n, ya sea por cuesti�n de tiempo o por que se a solicitado de esta manera directamente desde Control Escolar.
			\\
		\hline
			Precondiciones de entrada & Determinaci�n de cursos para el proximo periodo.\\

		\hline
			Relaci�n de productos \& Documentos.\\
				
		\hline
	\end{tabular}
\caption{Descripci�n de Instalaciones A1}
\end{table}




		%Archivo para insertar las actividades del proceso
\newpage
\subsubsection{Descripción de Roles}

%Exponer los actores que participan en el proceso y añadir una breve
%explicación de sus funciones en el proceso...

\begin{longtable}{|p{5cm}|p{11cm}|}%[H]
%\centering
%	\begin{tabular}{|p{5cm}|p{11cm}|}
		\hline
			\textbf{Rol} & \textbf{Descripción}\\ %Son los encabezados de la tabla 
		\hline\hline %A PARTIR DE AQUÍ SE EDITA
			{Alumno} & {Es la persona que ha tomado o va a tomar un curso, sus funciones son las siguientes:
		\begin{itemize}
			\item Hacer retroalimentaci\'on cada fin de curso.
		\end{itemize}

}\\ %dentro de las primeras llaves va el rol y despues dentro de las siguientes llaves va la descripcion del rol
		\hline
		{Coordinador} & {Persona encargada de gestionar los diferentes recursos y tareas para poder impartir los cursos en el periodo que se va a iniciar, sus funciones son:
			\begin{itemize}
				\item Gestionar la actualización, creación o cambio de los cursos y contenidos a ser impartidos por la CEC.
			\end{itemize}
			}\\
		\hline
		{Profesor} & {Es la persona que se encarga de impartir los cursos a los alumnos, así como también tiene la responsabilidad de definir los temarios de los cursos que va a impartir, sus funciones son:
		\begin{itemize}
			\item Sugerir nuevos cursos para impartir.
			\item Revisar el temario de cada curso que imparta.
			\item Realizar propuestas de nuevos cursos.
		\end{itemize}}\\\hline
				{UPIS} & {Es la unidad encargada de supervisar las labores realizadas por parte de la CEC, sus funciones son:
		\begin{itemize}
			\item Revisar la lista de cursos así como el contenido de cada uno de ellos para darles el seguimiento pertinente (actualizar, eliminar, restablecer).
		\end{itemize}}\\\hline
				{Academias vinculadas} & {Sectores de la ESCOM encargados de revisar los contenidos de los cursos que promueve la CEC, por el hecho de ser los especialistas en cada área que abordan dichos cursos:
		\begin{itemize}
			\item Revisar la lista de cursos así como el contenido de cada uno de ellos para darles el seguimiento pertinente (actualizar, eliminar, restablecer).
			\item Brindar recomendaciones acerca de los cursos.
			\item Proporcionar capital humano para impartir los cursos existentes.
		\end{itemize}}\\\hline
\caption{Identificación y descripción de roles}
\end{longtable}
		%Archivo para insertar las actividades del proceso
\newpage
\subsubsection{Descripción de Productos}
%Listar los productos (DOCUMENTOS) que cuenan con información del proceso

\begin{table}[H]
\centering
	\begin{tabular}{p{5cm}|p{11cm}}
		\hline
			\textbf{Producto} & \textbf{Descripción}\\ %Son los encabezados de la tabla 
		\hline\hline %A PARTIR DE AQUÍ SE EDITA
			{Temarios} & {Documento que se conforma de los cursos y contenidos que cada uno de ellos presenta, es la propuesta para ser expuesta a la comunidad interesada en los temas presentes.}\\ %dentro de las primeras llaves va el nombre del doc y despues dentro de las siguientes llaves va la descripcion del producto
		\hline
			%{producton} & {aquí va la descripción}\\
		%\hline
	\end{tabular}
\caption{Identificación y descripción de productos (Documentos)}
\end{table}

		%Archivo para insertar las actividades del proceso
	%En este archivo se encuentra la descripcion del proceso, este a su vez se sepera en otros archivos que se encuentran en la carpeta Descripcion en las diferentes secciones de la descripcion
% Aqui se colocaran las metricas para medir la calidad y extension del proceso
%Dentro de cada par de llaves despues de \footnotesize va lo que corresponde a cada columna de la tabla, hay 8 pares de llaves para
%cada renglon de la tabla, llenalos segun corresponda...
\subsection{Indicadores}
\begin{longtable}{|p{1.3cm}|p{1.4cm}|p{2cm}|p{2cm}|p{.8cm}|p{1.3cm}|p{2.2cm}|p{2cm}|}%[H]
%\centering
%	\begin{tabular}{|p{1.3cm}|p{1.4cm}|p{2cm}|p{2cm}|p{.8cm}|p{1.3cm}|p{2.2cm}|p{2cm}|}
		\hline
			\footnotesize\textbf{Nombre} & 
			\footnotesize\textbf{Objetivo} & 
			\footnotesize\textbf{Descripción} & 
			\footnotesize\textbf{Dimensión} & 
			\footnotesize\textbf{Tipo} & 
			\footnotesize\textbf{Formula} & 
			\footnotesize\textbf{Responsable} & 
			\footnotesize\textbf{Frecuencia de cálculo}\\
		\hline\hline%A PARTIR DE AQUI SE EDITA
			\footnotesize{Cantidad de alumnos que cursaron los cursos} & 
			\footnotesize{Obtener la cantidad de alumnos que participaron en el desarrollo de un curso} & 
			\footnotesize{Conocer y llevar una estadística de la cantidad de alumnos que se reinscriben periodo tras periodo} & 
			\footnotesize{Estadística} & 
			\footnotesize{De gestión} & 
			\footnotesize{%\%Estadistica\= \left(total de alumnos inscritos en el proceso \div total de alumnos regulares inscritos en el proceso\right)
			} & 
			\footnotesize{Coordinador} & 
			\footnotesize{Periodico (Duración del curso)}\\
		\hline\hline%A PARTIR DE AQUI SE EDITA
			\footnotesize{Cantidad de cursos aprobados} & 
			\footnotesize{Obtener la cantidad de cursos aprobados a ser promovidos} & 
			\footnotesize{Conocer y llevar una estadística de la cantidad de cursos que logran ser publicados} & 
			\footnotesize{Estadística} & 
			\footnotesize{De gestión} & 
			\footnotesize{%\%Estadistica\= \left(total de alumnos inscritos en el proceso \div total de alumnos regulares inscritos en el proceso\right)
			} & 
			\footnotesize{Coordinador} & 
			\footnotesize{Periodico (Duración del curso, propuesta de cursos)}\\
		\hline
\caption{Métricas de calidad y extensión del proceso}
\end{longtable}



%Aqu� se enlistan las reglas necesarias para el funcionamiento del proceso
%Lista de forma secuencial respecto a la forma de trabajar del proceso
%Extiende la tabla correspondiendo a la cantidad de Reglas del Proceso existentes en dicho Proceso
\subsection{Reglas del Proceso}
\begin{longtable}{p{5cm}|p{11cm}}%[H]
%\centering
%	\begin{tabular}{p{5cm}|p{11cm}}
		\hline
			\textbf{C�digo} & \textbf{Descripci�n}\\
		\hline\hline
			{{\color{blue}\textbf{RP1}}} & {El grupo debe tener un aula asignada para poder terminar el proceso.}\\ %en lugar de los puntos suspensivos debe ir la descripci�n de la regla
			\hline
			{{\color{blue}\textbf{RP2}}}  & {Un aula no puede estar asignada a dos grupos en un mismo horario.}\\
		\hline
			{{\color{blue}\textbf{RP3}}}  & {Los recursos auxiliares deben de estar preparados antes de que se d� inicio al curso.}\\
		\hline
			{{\color{blue}\textbf{RP4}}}  & {El grupo debe tener un aula asignada antes de que inicie el curso.}\\
		\hline
			{{\color{blue}\textbf{RP5}}}  & {}\\
		\hline
%	\end{tabular}
\caption{Lista de reglas del proceso}
\end{longtable}

%
%Aquí se enlistan las reglas necesarias para el funcionamiento del proceso
%Lista de forma secuencial respecto a la forma de trabajar del proceso
%Extiende la tabla correspondiendo a la cantidad de Reglas del Proceso existentes en dicho Proceso

\newpage
\subsection{Factores Críticos}
\begin{longtable}{p{5cm}|p{11cm}}%[H]
%\centering
%	\begin{tabular}{p{5cm}|p{11cm}}
		\hline
			\textbf{Factor Critico} & \textbf{Descripción}\\
		\hline\hline
			{} & {}\\
		\hline\hline
%	\end{tabular}
\caption{Lista de Factores Críticos}
\end{longtable}


%\newpage
\section{Glosario}

%Exponer los actores que participan en el proceso y añadir una breve
%explicación de sus funciones en el proceso...

\begin{longtable}{|p{5cm}|p{11cm}|}%[H]
%\centering
%	\begin{tabular}{|p{5cm}|p{11cm}|}
		\hline
			\textbf{Termino} & \textbf{Descripción}\\ %Son los encabezados de la tabla 
		\hline\hline %A PARTIR DE AQUÍ SE EDITA
			{} & {.}\\		
		\hline
%		{} & {}\\
		%\hline
		%\hline
%	\end{tabular}
\caption{Glosario}
\end{longtable}

%Indicar sólo la fecha de elaboración del Documento de administración
%del proceso...
\newpage
	%EJEMPLO
\subsection{Fecha de elaboración}
	\textbf{Fecha} %editar

%Firmas de elaboración, revisión y aprobación

%En este apartado se deberán asentar los nombres de los responsables
%de la elaboracion, revisión y aprobación del Documento administrativo
%del proceso, deberán incluirse las firmas autógrafas de los mismos.


	%Sólo coloquen su nombre
\subsection{Firmas de elaboración, revisión y aprobación}
	\textbf{Revisó y Aprobó: }\\
	\textbf{Redactó: }\\

