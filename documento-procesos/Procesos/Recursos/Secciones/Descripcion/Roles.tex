\newpage
\subsubsection{Descripci�n de Roles}

%Exponer los actores que participan en el proceso y a�adir una breve
%explicaci�n de sus funciones en el proceso...

\begin{longtable}{|p{5cm}|p{11cm}|}%[H]
%\centering
%	\begin{tabular}{|p{5cm}|p{11cm}|}
		\hline
			\textbf{Rol} & \textbf{Descripci�n}\\ %Son los encabezados de la tabla 
		\hline\hline %A PARTIR DE AQU� SE EDITA
			{Alumno} & {Es la persona que se encuentra inscrita y va a asistir a uno o varios cursos.

}\\ %dentro de las primeras llaves va el rol y despues dentro de las siguientes llaves va la descripcion del rol
		\hline
		{Encargado de Laboratorio} & {Es la persona que se encuentra en las aulas que cuentan con equipos de c�mputo, sus funciones son:
			\begin{itemize}
				\item Apoyar en la instalaci�n de software necesario en las computadoras.
			\end{itemize}
			}\\
		\hline
		{Profesor} & {Es la persona a cargo de impartir los cursos a los alumnos correspondientes, sus funciones son:
		\begin{itemize}
			\item Especifica los recursos auxiliares.
			\item Distribuye recursos auxiliares.
		\end{itemize}}\\\hline
\caption{Identificaci�n y descripci�n de roles}
\end{longtable}