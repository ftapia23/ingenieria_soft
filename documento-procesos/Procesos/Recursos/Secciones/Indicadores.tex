% Aqui se colocaran las metricas para medir la calidad y extension del proceso
%Dentro de cada par de llaves despues de \footnotesize va lo que corresponde a cada columna de la tabla, hay 8 pares de llaves para
%cada renglon de la tabla, llenalos segun corresponda...
\subsection{Indicadores}
\begin{longtable}{|p{1.3cm}|p{1.4cm}|p{2cm}|p{2cm}|p{.8cm}|p{1.3cm}|p{2.2cm}|p{2cm}|}%[H]
%\centering
%	\begin{tabular}{|p{1.3cm}|p{1.4cm}|p{2cm}|p{2cm}|p{.8cm}|p{1.3cm}|p{2.2cm}|p{2cm}|}
		\hline
			\footnotesize\textbf{Nombre} & 
			\footnotesize\textbf{Objetivo} & 
			\footnotesize\textbf{Descripci�n} & 
			\footnotesize\textbf{Dimensi�n} & 
			\footnotesize\textbf{Tipo} & 
			\footnotesize\textbf{Formula} & 
			\footnotesize\textbf{Responsable} & 
			\footnotesize\textbf{Frecuencia de c�lculo}\\
		\hline\hline%A PARTIR DE AQUI SE EDITA
			\footnotesize{Tiempo de Asignaci�n de aula} & 
			\footnotesize{Tener una estimaci�n del tiempo que tarda la actividad por curso.} & 
			\footnotesize{Se cuentan los d�as habiles en que se lleva a cabo la actividad de principio a fin.} & 
			\footnotesize{Estad�stica} & 
			\footnotesize{De administraci�n} & 
			\footnotesize{%\%Estadistica\= \left(total de alumnos inscritos en el proceso \div total de alumnos regulares inscritos en el proceso\right)
			} & 
			\footnotesize{Coordinador} & 
			\footnotesize{Cada vez que se asigne aula a un grupo del curso correspondiente.}\\
		\hline\hline%A PARTIR DE AQUI SE EDITA
			\footnotesize{Porcentaje de aulas asignadas} & 
			\footnotesize{Tener presente el avance de la asignaci�n de las aulas para los grupos 'abiertos'.} & 
			\footnotesize{Se considera como el 100\% los grupos 'abiertos', y el porcentaje va incrementandose respecto a los grupos que van teniendo un aula asignada, hasta que estas sean tantas como los grupos 'abiertos'.} & 
			\footnotesize{Estad�stica} & 
			\footnotesize{De administraci�n} & 
			\footnotesize{%\%Estadistica\= \left(total de alumnos inscritos en el proceso \div total de alumnos regulares inscritos en el proceso\right)
			} & 
			\footnotesize{Coordinador} & 
			\footnotesize{Durante la asignaci�n de aulas a los grupos 'abiertos'.}\\
		\hline\hline%A PARTIR DE AQUI SE EDITA
			\footnotesize{Tiempo de distribuci�n de los recursos auxiliares.} & 
			\footnotesize{Tener un tiempo para considerar al momento de administrar los recursos auxiliares, ya que estos deben estar disponibles antes del inicio de los cursos.} & 
			\footnotesize{Se cuentas los d�as habiles en los que se determinan los recursos auxiliares y estan listos para ser usados por el profesor o los alumnos.} & 
			\footnotesize{Estad�stica} & 
			\footnotesize{De administraci�n} & 
			\footnotesize{%\%Estadistica\= \left(total de alumnos inscritos en el proceso \div total de alumnos regulares inscritos en el proceso\right)
			} & 
			\footnotesize{Coordinador} & 
			\footnotesize{Cada vez que se determinen los recursos auxiliares para que se vaya teniendo nuevas referencias de tiempo.}\\
		\hline
\caption{M�tricas de calidad y extensi�n del proceso}
\end{longtable}
