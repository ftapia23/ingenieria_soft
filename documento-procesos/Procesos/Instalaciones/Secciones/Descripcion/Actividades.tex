\newpage
\subsubsection{Descripci�n de Actividades del Proceso}		%Descripcion de actividades del proceso



%%%%%%%%%%%%%%%%%%%%%%%%%%%%%%%%%%%% Actividades
\begin{table}[H]
\centering
	\begin{tabular}{p{5cm} p{11cm}}
		\hline
			\textbf{Proceso1-A1} & \textbf{Instalaciones}\\
		\hline\hline
			Descripci�n & 
			Una vez definida la cantidad de cursos que se impartiran durante el siguiente periodo, se establece cuantos salones y laboratorios, ser�n necesarios de acuerdo a la finalidad del curso, posteriormente se realiza una petici�n a Control Escolar de la disponibilidad de los salones y laboratorios de acuerdo a su uso de estos por parte de la ESCOM y otras instituciones que hagan uso de las instalaciones, en base a esta informaci�n se hace una primera selecci�n como propuesta para cada uno de los cursos, los cual se revisa y se le hacen modificaciones de acuerdo al horario deseado. 

			Una vez confirmado el horario en base a la disponibilidad de las instalaciones se procede a realizar un oficio solicitando a Control Escolar que se le facilite el uso de las �reas de trabajo seleccionadas para los cursos, una vez otorgado el permiso para el uso de las intalaciones por parte de la ESCOM se confirma la disponibilidad de las mismas para que se puedan publicar los horarios.

			Adicionalmente se guarda la informaci�n de los s�lones que no estan en uso durante los horarios establecidos, en caso de ser necesarios por razones que esten fuera del alcance de la administraci�n, ya sea por cuesti�n de tiempo o por que se a solicitado de esta manera directamente desde Control Escolar.
			\\
		\hline
			Precondiciones de entrada & Determinaci�n de cursos para el proximo periodo.\\

		\hline
			Relaci�n de productos \& Documentos.\\
				
		\hline
	\end{tabular}
\caption{Descripci�n de Instalaciones A1}
\end{table}




