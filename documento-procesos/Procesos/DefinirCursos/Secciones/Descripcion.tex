\subsection{Descripción del Proceso}
	La definición de cursos puede verse como una serie de actividades desencadenadas en tiempos variables que desembocan en un bien en común "formar cursos". Tenemos varios puntos de partida: propuesta de curso (por parte de profesores), historial de cursos, demanda de alumnos (personas que han tomado cursos anteriormente), toda está información es recabada al fin o bien sobre la marcha de los cursos a impartir, la cual es procesada por la CEC y coordinada por dos entidades: la UPIS y los departamentos académicos en vinculación (especialistas en el área). Al termino de está serie de actividades el material es propuesto y dispuesto a ser colocado en el mapa de cursos impartidos para su aceptación por la comunidad de alumnos con el fin de ser puesto en marcha.(Véase como un ciclo repetitivo de retro-alimentacion de materiales y mecanismos de educación)
\subsubsection{Criterios de Entrada}			%Criterios de entrada del proceso
%[Listar las condiciones que hacen que el proceso comience a trabajar].\\
	%Archivo para insertar los criterios de entrada
\subsubsection{Entradas}							%Entradas del proceso

\begin{table}[H]
\centering
	\begin{tabular}{p{5cm} p{11cm}}
		\hline
			\textbf{Entradas} & \textbf{Descripci�n}\\
		\hline
		\hline
		Horarios & Documento donde se especifica, las horas dedicadas por d�a de cada curso y su duraci�n aproximada de acuerdo al temario previsto. \\
		\hline
		 & .   \\	
		\hline
		 & .   \\
	    %		Copiar para agregar nueva fila con separando la entrada y la descripcion con &
	    %	\hline
	    %		Entrada N & Descripci�n de la entrada 2 \\	
	    %
		\hline
	\end{tabular}
\caption{Entradas de Nombre del Proceso}
\end{table}		%Archivo para insertar las entradas del proceso
\subsubsection{Criterios de Salida}			%Criterios de Salida del proceso
%[Listar las condiciones que hacen que el proceso comience a trabajar].\\

\begin{itemize}						%Inicia una lista con los criterios de salida
           \item ..
\end{itemize}	%Archivo para insertar los criterios de salida
\subsubsection{Salidas}

\begin{table}[H]
\centering
	\begin{tabular}{p{5cm} p{11cm}}
		\hline
			\textbf{Salidas} & \textbf{Descripción}\\
		\hline
		\hline
			Curso & Queda formalmente definido cuando quedan establecidos, revisados y en caso de ser necesario actualizados todos los temas que tienen que ser tratados a lo largo de la duraci\'on de este.
			\\
	
	    %		Copiar para agregar nueva fila con separando la entrada y la descripcion con &
	    %	\hline
	    %		Salida N & Descripción de la salida N \\	
	    %
		\hline	
	\end{tabular}
\caption{Entradas de Nombre del Proceso}
\end{table}
		%Archivo para insertar las salidas del proceso
\newpage
\subsubsection{Descripci�n de Actividades del Proceso}		%Descripcion de actividades del proceso



%%%%%%%%%%%%%%%%%%%%%%%%%%%%%%%%%%%% Actividades
\begin{table}[H]
\centering
	\begin{tabular}{p{5cm} p{11cm}}
		\hline
			\textbf{Proceso1-A1} & \textbf{Instalaciones}\\
		\hline\hline
			Descripci�n & 
			Una vez definida la cantidad de cursos que se impartiran durante el siguiente periodo, se establece cuantos salones y laboratorios, ser�n necesarios de acuerdo a la finalidad del curso, posteriormente se realiza una petici�n a Control Escolar de la disponibilidad de los salones y laboratorios de acuerdo a su uso de estos por parte de la ESCOM y otras instituciones que hagan uso de las instalaciones, en base a esta informaci�n se hace una primera selecci�n como propuesta para cada uno de los cursos, los cual se revisa y se le hacen modificaciones de acuerdo al horario deseado. 

			Una vez confirmado el horario en base a la disponibilidad de las instalaciones se procede a realizar un oficio solicitando a Control Escolar que se le facilite el uso de las �reas de trabajo seleccionadas para los cursos, una vez otorgado el permiso para el uso de las intalaciones por parte de la ESCOM se confirma la disponibilidad de las mismas para que se puedan publicar los horarios.

			Adicionalmente se guarda la informaci�n de los s�lones que no estan en uso durante los horarios establecidos, en caso de ser necesarios por razones que esten fuera del alcance de la administraci�n, ya sea por cuesti�n de tiempo o por que se a solicitado de esta manera directamente desde Control Escolar.
			\\
		\hline
			Precondiciones de entrada & Determinaci�n de cursos para el proximo periodo.\\

		\hline
			Relaci�n de productos \& Documentos.\\
				
		\hline
	\end{tabular}
\caption{Descripci�n de Instalaciones A1}
\end{table}




		%Archivo para insertar las actividades del proceso
\newpage
\subsubsection{Descripción de Roles}

%Exponer los actores que participan en el proceso y añadir una breve
%explicación de sus funciones en el proceso...

\begin{longtable}{|p{5cm}|p{11cm}|}%[H]
%\centering
%	\begin{tabular}{|p{5cm}|p{11cm}|}
		\hline
			\textbf{Rol} & \textbf{Descripción}\\ %Son los encabezados de la tabla 
		\hline\hline %A PARTIR DE AQUÍ SE EDITA
			{Alumno} & {Es la persona que ha tomado o va a tomar un curso, sus funciones son las siguientes:
		\begin{itemize}
			\item Hacer retroalimentaci\'on cada fin de curso.
		\end{itemize}

}\\ %dentro de las primeras llaves va el rol y despues dentro de las siguientes llaves va la descripcion del rol
		\hline
		{Coordinador} & {Persona encargada de gestionar los diferentes recursos y tareas para poder impartir los cursos en el periodo que se va a iniciar, sus funciones son:
			\begin{itemize}
				\item Gestionar la actualización, creación o cambio de los cursos y contenidos a ser impartidos por la CEC.
			\end{itemize}
			}\\
		\hline
		{Profesor} & {Es la persona que se encarga de impartir los cursos a los alumnos, así como también tiene la responsabilidad de definir los temarios de los cursos que va a impartir, sus funciones son:
		\begin{itemize}
			\item Sugerir nuevos cursos para impartir.
			\item Revisar el temario de cada curso que imparta.
			\item Realizar propuestas de nuevos cursos.
		\end{itemize}}\\\hline
				{UPIS} & {Es la unidad encargada de supervisar las labores realizadas por parte de la CEC, sus funciones son:
		\begin{itemize}
			\item Revisar la lista de cursos así como el contenido de cada uno de ellos para darles el seguimiento pertinente (actualizar, eliminar, restablecer).
		\end{itemize}}\\\hline
				{Academias vinculadas} & {Sectores de la ESCOM encargados de revisar los contenidos de los cursos que promueve la CEC, por el hecho de ser los especialistas en cada área que abordan dichos cursos:
		\begin{itemize}
			\item Revisar la lista de cursos así como el contenido de cada uno de ellos para darles el seguimiento pertinente (actualizar, eliminar, restablecer).
			\item Brindar recomendaciones acerca de los cursos.
			\item Proporcionar capital humano para impartir los cursos existentes.
		\end{itemize}}\\\hline
\caption{Identificación y descripción de roles}
\end{longtable}
		%Archivo para insertar las actividades del proceso
\newpage
\subsubsection{Descripción de Productos}
%Listar los productos (DOCUMENTOS) que cuenan con información del proceso

\begin{table}[H]
\centering
	\begin{tabular}{p{5cm}|p{11cm}}
		\hline
			\textbf{Producto} & \textbf{Descripción}\\ %Son los encabezados de la tabla 
		\hline\hline %A PARTIR DE AQUÍ SE EDITA
			{Temarios} & {Documento que se conforma de los cursos y contenidos que cada uno de ellos presenta, es la propuesta para ser expuesta a la comunidad interesada en los temas presentes.}\\ %dentro de las primeras llaves va el nombre del doc y despues dentro de las siguientes llaves va la descripcion del producto
		\hline
			%{producton} & {aquí va la descripción}\\
		%\hline
	\end{tabular}
\caption{Identificación y descripción de productos (Documentos)}
\end{table}

		%Archivo para insertar las actividades del proceso

