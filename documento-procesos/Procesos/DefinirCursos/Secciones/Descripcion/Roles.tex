\newpage
\subsubsection{Descripción de Roles}

%Exponer los actores que participan en el proceso y añadir una breve
%explicación de sus funciones en el proceso...

\begin{longtable}{|p{5cm}|p{11cm}|}%[H]
%\centering
%	\begin{tabular}{|p{5cm}|p{11cm}|}
		\hline
			\textbf{Rol} & \textbf{Descripción}\\ %Son los encabezados de la tabla 
		\hline\hline %A PARTIR DE AQUÍ SE EDITA
			{Alumno} & {Es la persona que ha tomado o va a tomar un curso, sus funciones son las siguientes:
		\begin{itemize}
			\item Hacer retroalimentaci\'on cada fin de curso.
		\end{itemize}

}\\ %dentro de las primeras llaves va el rol y despues dentro de las siguientes llaves va la descripcion del rol
		\hline
		{Coordinador} & {Persona encargada de gestionar los diferentes recursos y tareas para poder impartir los cursos en el periodo que se va a iniciar, sus funciones son:
			\begin{itemize}
				\item Gestionar la actualización, creación o cambio de los cursos y contenidos a ser impartidos por la CEC.
			\end{itemize}
			}\\
		\hline
		{Profesor} & {Es la persona que se encarga de impartir los cursos a los alumnos, así como también tiene la responsabilidad de definir los temarios de los cursos que va a impartir, sus funciones son:
		\begin{itemize}
			\item Sugerir nuevos cursos para impartir.
			\item Revisar el temario de cada curso que imparta.
			\item Realizar propuestas de nuevos cursos.
		\end{itemize}}\\\hline
				{UPIS} & {Es la unidad encargada de supervisar las labores realizadas por parte de la CEC, sus funciones son:
		\begin{itemize}
			\item Revisar la lista de cursos así como el contenido de cada uno de ellos para darles el seguimiento pertinente (actualizar, eliminar, restablecer).
		\end{itemize}}\\\hline
				{Academias vinculadas} & {Sectores de la ESCOM encargados de revisar los contenidos de los cursos que promueve la CEC, por el hecho de ser los especialistas en cada área que abordan dichos cursos:
		\begin{itemize}
			\item Revisar la lista de cursos así como el contenido de cada uno de ellos para darles el seguimiento pertinente (actualizar, eliminar, restablecer).
			\item Brindar recomendaciones acerca de los cursos.
			\item Proporcionar capital humano para impartir los cursos existentes.
		\end{itemize}}\\\hline
\caption{Identificación y descripción de roles}
\end{longtable}
