\subsubsection{Descripción de Actividades del Proceso}		%Descripcion de actividades del proceso
%%%%%%%%%%%%%%%%%%%%%%%%%%%%%%%%%%%% Actividades
\begin{table}[H]
\centering
	\begin{tabular}{p{5cm} p{11cm}}
		\hline\hline
			\rowcolor{LightBlue2} \textbf{P1-A1} & \textbf{Retro-alimentación de alumnos}\\
		\hline\hline
			Descripción & Se le pide a los alumnos que escriban una retroalimentación y se toman en consideración sus opiniones con el fin de saber si el curso cumplió con sus expectativas y cubrió todo el temario  para definir si debe de seguir siendo impartido, actualizado o descartado en el próximo periodo.\\
		\hline
			Precondiciones de entrada & Encontrarse al final de cada curso impartido.\\					
		\hline
			Relación de productos & Hoja de retro-alimentación por parte de los alumnos.\\
		\hline
	\end{tabular}
\caption{Descripción de retro-alimentación de alumnos}
\end{table}

\begin{table}[H]
	\centering
	\begin{tabular}{p{5cm} p{11cm}}
		\hline
		\rowcolor{LightBlue2} \textbf{Proceso1-A2} & \textbf{Propuesta de curso}\\
		\hline\hline
		Descripción & El profesor presenta una propuesta para impartir un curso nuevo que puede ser de interés para los estudiantes o personas externas a la comunidad, el encargado analiza la propuesta y el temario que se pretende impartir, si es aceptado se pone la lista de los cursos que serán impartidos el próximo periodo.\\
		\hline
		Precondiciones de entrada & 	
		\begin{itemize}
			\item El curso no se había impartido antes.
			\item Debe de contar con un temario que abarque el tiempo que dura un curso.
		\end{itemize}
		 \\					
		\hline
		Relación de productos & Temario del curso\\
		\hline
	\end{tabular}
	\caption{Descripción de Propuesta de curso}
\end{table}


\begin{table}[H]
	\centering
	\begin{tabular}{p{5cm} p{11cm}}
		\hline
		\rowcolor{LightBlue2} \textbf{Proceso1-A3} & \textbf{Definición de cursos}\\
		\hline\hline
		Descripción & Con base en la retro-alimentación y las sugerencias de nuevos cursos por parte de los profesores el coordinador en apoyo de las entidades externas realiza un análisis de cuales deben de ser los cursos que se deben de impartir el próximo periodo teniendo en cuenta los siguientes puntos:
		\begin{itemize}
			\item Disponibilidad de recursos(instalaciones, docentes).
			\item Demanda de cada curso.
		\end{itemize}
		 \\
		\hline
		Precondiciones de entrada & Contar con la retro-alimentación de los alumnos y las propuestas de cursos nuevos por parte de los profesores. \\				
		\hline
		Relación de productos & Asignación de recursos\\
		\hline
	\end{tabular}
	\caption{Descripción de la Definición de cursos}
\end{table}


\begin{table}[H]
	\centering
	\begin{tabular}{p{5cm} p{11cm}}
		\hline
		\rowcolor{LightBlue2} \textbf{Proceso1-A4} & \textbf{Definición de temarios}\\
		\hline\hline
		Descripción & El profesor asignado para impartir un curso revisa detalladamente el temario con el fin de ver si es necesario hacer una corrección en los temas, de ser necesario agregar un nuevo tema o de sustituir uno o más con la informaci\'on m\'as reciente, as\'i como tambi\'en debe de hacer una aproximaci\'on de cuanto tiempo debe dedicarle a cada tema con el fin de abarcar todo el temario. Est\'a actividad debe de realizarse para cada uno de los cursos que se van a impartir. En caso de ser un curso nuevo el temario solo seria revisado con el fin de medir cuanto tiempo se debe dedicar en promedio a cada tema.\\ 
		\hline
		Precondiciones de entrada & 
		\begin{itemize}
			\item Asignaci\'on de profesores
			\item Definici\'on de cursos a impartir
		\end{itemize}
		\\					
		\hline
		Relación de productos & Lista de definici\'on de cursos\\
		\hline
	\end{tabular}
	\caption{Descripción de la definición de temarios}
\end{table}

\begin{table}[H]
	\centering
	\begin{tabular}{p{5cm} p{11cm}}
		\hline
		\rowcolor{LightBlue2} \textbf{Proceso1-A5} & \textbf{Aprobación de cursos}\\
		\hline\hline
		Descripción & La UPIS en apoyo de las academias vinculadas de la ESCOM trabajan en conjunto, para aprobar los nuevos cursos así como los cursos existentes con el fin de darle seguimiento al proceso de definición de cursos.\\
		\hline
		Precondiciones de entrada & 
		\begin{itemize}
			\item Contar con cursos a impartir.
		\end{itemize}
		\\					
		\hline
		Relación de productos & Lista de cursos\\
		\hline
	\end{tabular}
	\caption{Descripción de la aprobación de cursos}
\end{table}
