\subsubsection{Actividades del Proceso}		%Descripcion de actividades del proceso
%%%%%%%%%%%%%%%%%%%%%%%%%%%%%%%%%%%% Actividades
\begin{table}[H]
\centering
	\begin{tabular}{p{3.5cm} p{12.5cm}}
		\hline\hline
			\rowcolor{LightBlue2} \textbf{P1-A1} & \textbf{Retro-alimentación de alumnos}\\
		\hline\hline
			Descripción & Se solicita a los alumnos que lleven acabo una retroalimentación, para tomar en consideración sus opiniones con el fin de saber si el curso cumplió con sus expectativas y cubrió el temario en su totalidad para definir si debe de seguir siendo impartido, actualizado o descartado en el próximo periodo.\\	
		\hline	\hline		
			\multicolumn{2}{|c|}{\textbf{Factores críticos:}}\\
		\hline\hline
			{El alumno deberá:}&
			\begin{enumerate}
				\item Responder a una encuesta con el objeto de aumentar la calidad en el servicio de los cursos impartidos en la CEC.
			\end{enumerate}\\
			{El coordinador responsable deberá:}&
			\begin{enumerate}
				\item Identificar los factores representativos de la retro-alimentación para la evolución de los cursos existentes.
				\item Designar a profesores que con ayuda de su respectiva academia vinculada, atiendan las peticiones de los resultados arrojados de los instrumentos evaluados según la retro-alimentación.
				\item Cualificar las peticiones expuestas por los alumnos interesados en el progreso de la CEC.
			\end{enumerate}\\
		\hline
			Relación de productos & Hoja de retro-alimentación por parte de los alumnos.\\
		\hline
	\end{tabular}
\end{table}

\begin{table}[H]
	\centering
	\begin{tabular}{p{3.5cm} p{12.5cm}}
		\hline
		\rowcolor{LightBlue2} \textbf{P1-A2} & \textbf{Propuesta de curso}\\
		\hline\hline
		Descripción & El profesor perteneciente a la CEC presenta una propuesta para impartir un curso nuevo que puede ser de interés para los estudiantes o personas externas a la comunidad, el encargado analiza la propuesta y el temario que se pretende impartir, si es aceptado se pone la lista de los cursos que serán implementados el próximo periodo.\\			
		\hline	\hline		
			\multicolumn{2}{|c|}{\textbf{Factores críticos:}}\\
		\hline\hline
			{El profesor que propone un nuevo curso deberá:}&
			\begin{enumerate}
				\item Realizar la descripción detallada del curso a impartir, así como los objetivos y el impacto que este pudiese tener.
				\item Desarrollar el temario correspondiente al curso que se pretende implantar.
				\item Entregar la propuesta al coordinador en turno de la CEC y su academia vinculada para ser evaluada.
			\end{enumerate}\\
		\hline
		Relación de productos & Temario del curso, Encuesta de retro-alimentación\\
		\hline
	\end{tabular}
\end{table}


\begin{table}[H]
	\centering
	\begin{tabular}{p{3.5cm} p{12.5cm}}
		\hline
		\rowcolor{LightBlue2} \textbf{P1-A3} & \textbf{Definición de cursos}\\
		\hline\hline
		Descripción & Con base en la retro-alimentación y las sugerencias de nuevos cursos por parte de los profesores el coordinador en apoyo de las academias vinculadas realiza un análisis de cuales deben de ser los cursos que se deben de impartir el próximo periodo teniendo en cuenta los siguientes puntos:
		\begin{itemize}
			\item Disponibilidad de recursos(instalaciones, docentes).
			\item Demanda de cada curso.
		\end{itemize}\\
		 \hline\hline		
			\multicolumn{2}{|c|}{\textbf{Factores críticos:}}\\
		\hline\hline
			{El coordinador deberá:}&
			\begin{enumerate}
				\item Evaluar las propuestas y cruzar la información con los datos obtenidos de los indicadores de retro-alimentación. 
				\item Aprobar los cursos que son propuestos por parte de los profesores.
				\item Verificar la actualización de los cursos existentes.
			\end{enumerate}\\
			{La academia vinculada deberá:}&
			\begin{enumerate}
				\item Revisar y actualizar los cursos existentes.
				\item Aprobar los cursos propuestos por los profesores que pertenecen a la academia.
			\end{enumerate}\\
		\hline
		Relación de productos & Asignación de recursos\\
		\hline
	\end{tabular}
\end{table}

\begin{table}[H]
	\centering
	\begin{tabular}{p{3.5cm} p{12.5cm}}
		\hline
		\rowcolor{LightBlue2} \textbf{P1-A5} & \textbf{Aprobación de cursos}\\
		\hline\hline
		Descripción & La UPIS en apoyo de las academias vinculadas de la ESCOM trabajan en conjunto, para aprobar los nuevos cursos así como los cursos existentes con el fin de darle seguimiento al proceso de definición de cursos.\\
		\hline	\hline		
			\multicolumn{2}{|c|}{\textbf{Factores críticos:}}\\
		\hline\hline
			{La UPIS deberá:}&
			\begin{enumerate}
				\item Aprobar los cursos en planificación y los que se encuentran en actualización con la finalidad de que los cursos a impartir son los más adecuados.
			\end{enumerate}\\
		\hline
		Relación de productos & Lista de cursos\\
		\hline
	\end{tabular}
\end{table}

\begin{table}[H]
	\centering
	\begin{tabular}{p{3.5cm} p{12.5cm}}
		\hline
		\rowcolor{LightBlue2} \textbf{Proceso1-A4} & \textbf{Definición de temarios}\\
		\hline\hline
		Descripción & El profesor asignado para impartir un curso es el encargado de revisar detalladamente el temario con el fin de ver si es necesario hacer una corrección en los temas, de ser necesario agregar un nuevo tema o de sustituir uno o más con la información más reciente, así como también debe de hacer una aproximación de cuanto tiempo debe dedicarle a cada tema con el objeto de abarcar todo el temario. Está actividad debe de realizarse para cada uno de los cursos que se van a impartir. En caso de ser un curso nuevo el temario solo seria revisado con el fin de medir cuanto tiempo se debe dedicar en promedio a cada tema y de verificar que los temas cumplan con el objetivo del curso.\\	
		\hline	\hline		
			\multicolumn{2}{|c|}{\textbf{Factores críticos:}}\\
		\hline\hline
			{El responsable deberá:}&
			\begin{enumerate}
				\item 
				\item 
			\end{enumerate}\\		
		\hline
		Relación de productos & Lista de definici\'on de cursos\\
		\hline
	\end{tabular}
\end{table}