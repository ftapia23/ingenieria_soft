\newpage
\subsubsection{Descripción de Actividades del Proceso}		%Descripcion de actividades del proceso



%%%%%%%%%%%%%%%%%%%%%%%%%%%%%%%%%%%% Actividades
\begin{table}[H]
\centering
	\begin{tabular}{p{5cm} p{11cm}}
		\hline
			\textbf{Proceso1-A1} & \textbf{Retroalimentaci\'on de alumnos}\\
		\hline\hline
			Descripción & Se le pide a los alumnos que escriban una retroalimentaci\'on y se toman en consideracion sus opiniones con el fin de saber si el curso cumpli\'o con sus expectativas, si se cubri\'o todo el temario y de saber si el curso debe de seguir siendo impartido, actualizado o descartado para impartirse en el pr\'oximo periodo.\\
		\hline
			Precondiciones de entrada & Estar en la etapa de cierre del curso.\\					
		\hline
			Relación de productos & Hoja de retroalimentaci\'on por parte de los alumnos.\\
		\hline
	\end{tabular}
\caption{Descripción de Retroalimentaci\'on A1}
\end{table}

\begin{table}[H]
	\centering
	\begin{tabular}{p{5cm} p{11cm}}
		\hline
		\textbf{Proceso1-A2} & \textbf{Propuesta de curso}\\
		\hline\hline
		Descripción & El profesor presenta una propuesta para impartir un curso nuevo que puede ser de interes para los estudiantes o p\'ublico en general, el encargado analiza la propuesta y el temario que se pretende impartir, si es aceptado se pone la lista de los cursos que ser\'an impartidos el pr\'oximo periodo.\\
		\hline
		Precondiciones de entrada & 
		
		\begin{itemize}
			\item El curso no se habia impartido antes
			\item Debe de contar con un temario que abarque el tiempo que dura un curso
		\end{itemize}
		
		 \\					
		\hline
		Relación de productos & Temario del curso\\
		\hline
	\end{tabular}
	\caption{Descripción de  A2}
\end{table}


\begin{table}[H]
	\centering
	\begin{tabular}{p{5cm} p{11cm}}
		\hline
		\textbf{Proceso1-A3} & \textbf{Definici\'on de cursos}\\
		\hline\hline
		Descripción & Con base en la retroalimentaci\'on y las sugerencias de nuevos cursos por parte de los profesores el coordinador realiza un an\'alisis de cuales deben de ser los cursos que se deben de impartir el pr\'oximo periodo teniendo en cuenta los siguientes puntos:
		\begin{itemize}
			\item Disponibilidad de recursos
			\item Demanda de cada curso
		\end{itemize}
		 \\
		\hline
		Precondiciones de entrada & Contar con la retroalimentaci\'on de los alumnos y las propuestas de cursos nuevos por parte de los profesores. \\					
		\hline
		Relación de productos & Asignaci\'on de recursos\\
		\hline
	\end{tabular}
	\caption{Descripción de Definici\'on de cursos A3}
\end{table}


\begin{table}[H]
	\centering
	\begin{tabular}{p{5cm} p{11cm}}
		\hline
		\textbf{Proceso1-A4} & \textbf{Definici\'on de temarios}\\
		\hline\hline
		Descripción & El profesor asignado para impartir un curso revisa detalladamente el temario con el fin de ver si es necesario hacer una correci\'on en los temas, de ser necesario agregar un nuevo tema o de sustituir uno o m\'as con la informaci\'on m\'as reciente, as\'i como tambi\'en debe de hacer una aproximaci\'on de cuanto tiempo debe dedicarle a cada tema con el fin de abarcar todo el temario. Est\'a actividad debe de realizarse para cada uno de los cursos que se van a impartir. En caso de ser un curso nuevo el temario solo ser\'a revisado con el fin de medir cuanto tiempo se debe dedicar en promedio a cada tema.\\ 
		\hline
		Precondiciones de entrada & 
		\begin{itemize}
			\item Asignaci\'on de profesores
			\item Definici\'on de cursos a impartir
		\end{itemize}
		\\					
		\hline
		Relación de productos & Lista de definici\'on de cursos\\
		\hline
	\end{tabular}
	\caption{Descripción de Definici\'on de temarios A4}
\end{table}
