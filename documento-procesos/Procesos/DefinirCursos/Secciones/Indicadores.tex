% Aqui se colocaran las metricas para medir la calidad y extension del proceso
%Dentro de cada par de llaves despues de \footnotesize va lo que corresponde a cada columna de la tabla, hay 8 pares de llaves para
%cada renglon de la tabla, llenalos segun corresponda...
\subsection{Indicadores}
\begin{longtable}{|p{1.3cm}|p{1.4cm}|p{2cm}|p{2cm}|p{.8cm}|p{1.3cm}|p{2.2cm}|p{2cm}|}%[H]
%\centering
%	\begin{tabular}{|p{1.3cm}|p{1.4cm}|p{2cm}|p{2cm}|p{.8cm}|p{1.3cm}|p{2.2cm}|p{2cm}|}
		\hline
			\footnotesize\textbf{Nombre} & 
			\footnotesize\textbf{Objetivo} & 
			\footnotesize\textbf{Descripción} & 
			\footnotesize\textbf{Dimensión} & 
			\footnotesize\textbf{Tipo} & 
			\footnotesize\textbf{Formula} & 
			\footnotesize\textbf{Responsable} & 
			\footnotesize\textbf{Frecuencia de cálculo}\\
		\hline\hline%A PARTIR DE AQUI SE EDITA
			\footnotesize{Cantidad de alumnos que cursaron los cursos} & 
			\footnotesize{Obtener la cantidad de alumnos que participaron en el desarrollo de un curso} & 
			\footnotesize{Conocer y llevar una estadística de la cantidad de alumnos que se reinscriben periodo tras periodo} & 
			\footnotesize{Estadística} & 
			\footnotesize{De gestión} & 
			\footnotesize{%\%Estadistica\= \left(total de alumnos inscritos en el proceso \div total de alumnos regulares inscritos en el proceso\right)
			} & 
			\footnotesize{Coordinador} & 
			\footnotesize{Periodico (Duración del curso)}\\
		\hline\hline%A PARTIR DE AQUI SE EDITA
			\footnotesize{Cantidad de cursos aprobados} & 
			\footnotesize{Obtener la cantidad de cursos aprobados a ser promovidos} & 
			\footnotesize{Conocer y llevar una estadística de la cantidad de cursos que logran ser publicados} & 
			\footnotesize{Estadística} & 
			\footnotesize{De gestión} & 
			\footnotesize{%\%Estadistica\= \left(total de alumnos inscritos en el proceso \div total de alumnos regulares inscritos en el proceso\right)
			} & 
			\footnotesize{Coordinador} & 
			\footnotesize{Periodico (Duración del curso, propuesta de cursos)}\\
		\hline
\caption{Métricas de calidad y extensión del proceso}
\end{longtable}

