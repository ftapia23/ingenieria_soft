%Aquí se enlistan las reglas necesarias para el funcionamiento del proceso
%Lista de forma secuencial respecto a la forma de trabajar del proceso
%Extiende la tabla correspondiendo a la cantidad de Reglas del Proceso existentes en dicho Proceso
\subsection{Reglas del Proceso}
\begin{longtable}{p{5cm}|p{11cm}}%[H]
%\centering
%	\begin{tabular}{p{5cm}|p{11cm}}
		\hline
			\textbf{Código} & \textbf{Descripción}\\
		\hline\hline
			{{\color{blue}\textbf{RP1}}} & {Los cursos no pueden ser publicados sin antes ser aprobados por los coordinadores.}\\ %en lugar de los puntos suspensivos debe ir la descripción de la regla
			\hline
			{{\color{blue}\textbf{RP2}}}  & {La retro-alimentación de cursos existentes debe ser realizada por alumnos que pertenecieron a los cursos con anterioridad.}\\
		\hline
			{{\color{blue}\textbf{RP3}}}  & {Las propuestas de docentes, deben ser expuestas por profesores inscritos en la CEC.}\\
		\hline
			{{\color{blue}\textbf{RP4}}}  & {No pueden existir dos cursos idénticos que se encuentren en estatus que les permita ser publicados.}\\
		\hline
%	\end{tabular}
\caption{Lista de reglas del proceso}
\end{longtable}

