\newpage
\subsubsection{Descripción de Productos}
%Listar los productos (DOCUMENTOS) que cuenan con información del proceso

\begin{table}[H]
\centering
	\begin{tabular}{p{5cm}|p{11cm}}
		\hline
			\textbf{Producto} & \textbf{Descripción}\\ %Son los encabezados de la tabla 
		\hline\hline %A PARTIR DE AQUÍ SE EDITA
			{Temarios} & {Documento que se conforma de los cursos y contenidos que cada uno de ellos presenta, es la propuesta para ser expuesta a la comunidad interesada en los temas presentes.}\\ %dentro de las primeras llaves va el nombre del doc y despues dentro de las siguientes llaves va la descripcion del producto
			\hline
			{Lista de cursos} & {Documento en el cual se enlistan todos los cursos que han sido aprobados para ser impartidos a los alumnos.}\\ 
			\hline
			{Propuesta de curso} & {Documento que contiene de manera detallada la propuesta de impartir un curso nuevo, este documento debe de traer una justificación, un temario que abarque el tiempo mínimo que debe de durar un curso junto con una breve descripción de lo que trata cada tema y una lista de los recursos necesarios para poder impartir el curso.}\\ 
			%{producton} & {aquí va la descripción}\\
		%\hline
	\end{tabular}
\caption{Identificación y descripción de productos (Documentos)}
\end{table}

