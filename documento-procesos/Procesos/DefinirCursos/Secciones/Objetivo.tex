
\subsection{Objetivos del Proceso} 	%Objetivos del proceso
%[Definir los objetivos del proceso, en el contexto del funcionamiento del sistema (máximo 20 líneas). Y que responde a la siguiente
%pregunta ¿Cuál es la razón de ser este proceso? en otras palabras que hace y para que lo hace. 1 Objetivo general, n objetivos
% específicos].

\subsubsection{Objetivo General}	%Objetivos generales
		Crear el mecanismo para la actualización y creación de nuevos cursos, con la finalidad de ser impartidos en la CEC dirigidos a toda entidad(alumnos, empresas) que tenga la necesidad de adquirir conocimientos en ciertas áreas de interés, del sector de la Ingeniería en sistemas computacionales y ciencias complementarias (llamadas ciencias básicas: matemáticas y física). Así logrando el aumento de la calidad en el servicio prestado.
\subsubsection{Objetivos Específicos}	%Objetivos especifico
\begin{enumerate}[I] 			%Empieza una lista numerada con los objetivos especificos.
	\item Integrar cursos que sean sugeridos por algún profesor o por la alta demanda que estos tengan durante su implantación, con el fin de aumentar la diversidad educativa y el fomento de una actualización constante.
	\item Actualizar los cursos existentes en la CEC periódicamente con el objeto de incrementar su nivel educativo.
	\item Eliminar si fuese el caso cursos que pierdan el interés de los solicitantes o por la actualización en modelos educativos.
\end{enumerate}

