\newpage
\subsubsection{Descripción de Roles}

%Exponer los actores que participan en el proceso y añadir una breve
%explicación de sus funciones en el proceso...

\begin{longtable}{|p{5cm}|p{11cm}|}%[H]
%\centering
%	\begin{tabular}{|p{5cm}|p{11cm}|}
		\hline
			\textbf{Rol} & \textbf{Descripción}\\ %Son los encabezados de la tabla 
		\hline\hline %A PARTIR DE AQUÍ SE EDITA
			{Alumno} & {Es la persona que ha tomado o va a tomar un curso, sus funciones son las siguientes:
		\begin{itemize}
			\item Hacer retroalimentaci\'on cada fin de curso
			\item Cubrir las horas necesarias para poder acreeditar el curso 
		\end{itemize}

}\\ %dentro de las primeras llaves va el rol y despues dentro de las siguientes llaves va la descripcion del rol
		\hline
		{Coordinador} & {Persona encargada de gestionar los diferentes recursos y tareas para poder impartir los cursos en el periodo que se va a iniciar, sus funciones son:
			\begin{itemize}
				\item Gestionar a los profesores
				\item Gestionar las instalaciones que se van a utilizar
				\item Planear y gestionar cursos
			\end{itemize}
			}\\
		\hline
		{Profesor} & {Es la persona que se encarga de impartir los cursos a los alumnos, así como tambien tiene la responsabilidad de definir los temarios de los cursos que va a impartir, sus funciones son:
		\begin{itemize}
			\item Impartir los temas a los alumnos de cada curso que tenga asignado
			\item Sugerir nuevos cursos para impartir
			\item Revisar el temario de cada curso que imparta
			\item Evaluar a los alumnos
		\end{itemize}}\\
\caption{Identificación y descripción de roles}
\end{longtable}
